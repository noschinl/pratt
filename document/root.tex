\documentclass[11pt,a4paper]{article}
\usepackage{isabelle,isabellesym}

% further packages required for unusual symbols (see also
% isabellesym.sty), use only when needed

\usepackage{amssymb}
  %for \<leadsto>, \<box>, \<diamond>, \<sqsupset>, \<mho>, \<Join>,
  %\<lhd>, \<lesssim>, \<greatersim>, \<lessapprox>, \<greaterapprox>,
  %\<triangleq>, \<yen>, \<lozenge>

%\usepackage{eurosym}
  %for \<euro>

%\usepackage[only,bigsqcap]{stmaryrd}
  %for \<Sqinter>

%\usepackage{eufrak}
  %for \<AA> ... \<ZZ>, \<aa> ... \<zz> (also included in amssymb)

%\usepackage{textcomp}
  %for \<onequarter>, \<onehalf>, \<threequarters>, \<degree>, \<cent>,
  %\<currency>

% this should be the last package used
\usepackage{pdfsetup}

% urls in roman style, theory text in math-similar italics
\urlstyle{rm}
\isabellestyle{it}

% for uniform font size
%\renewcommand{\isastyle}{\isastyleminor}


\begin{document}

\title{Pratt's Primality Certificates \\ (Isabelle Formalisation)}
\author{By Simon Wimmer and Lars Noschinski}
\maketitle

\begin{abstract}
  In 1975, Pratt introduced a proof system for certifying primes
  \cite{pratt1975certificate}.
  He showed that a number $p$ is prime iff a primality certificate for $p$ exists.
  By showing a logarithmic upper bound on the length of the certificates in size of the prime number,
  he concluded that the decision problem for prime numbers is in NP.
  This work formalizes Pratt's proof system and shows the results mentioned before.  
  The proofs are based on Lehmer's theorem, a converse of Fermat's little theorem.
  A proof of Lehmer's theorem constitutes the first part of this work.
  As a side product we formalize
  some properties of the Euler $\phi$-function,
  the notion of the order of an element of a group,
  and the cyclicity of the multiplicative group of a finite field.
\end{abstract}

\tableofcontents

Section \ref{sec:euler-phi} to \ref{sec:number-roots} formalize some basic number-theoretic
and algebraic properties: Euler's $\phi$-function, the order of an element of a group
and an upper bound of the number of roots of a polynomial. Section \ref{sec:mult-group}
combines these results to prove that the multiplicative group of a finite field is cyclic.
Based on that, Section \ref{sec:lehmer} formalizes an extended version of Lehmer's Theorem,
which gives us necessary and sufficient conditions to decide whether a number is prime.
Finally, Section \ref{sec:pratt} proves soundness and completeness for Pratt's Primality
Certificates. Moreover, it also shows an upper bound for the length of the certificates.

% sane default for proof documents
\parindent 0pt\parskip 0.5ex

% generated text of all theories
\input{session}

\nocite{*}

% optional bibliography
\bibliographystyle{abbrv}
\bibliography{root}

\end{document}

%%% Local Variables:
%%% mode: latex
%%% TeX-master: t
%%% End\dots
